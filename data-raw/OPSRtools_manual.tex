%%%%%

\inputencoding{utf8}
\HeaderA{OPSRtools-package}{OPSRtools: Tools for OPSR}{OPSRtools.Rdash.package}
\aliasA{OPSRtools}{OPSRtools-package}{OPSRtools}
%
\begin{Description}
OPSR estimates ordered probit switching regression models - a Heckman type endogenous switching regression model with an ordinal selection/treatment and continuous outcomes. OPSRtools contains estimation and post-estimation routines such as step-wise model selection, k-fold cross-validation, treatment effect computations and some visualizations.
\end{Description}
%
\begin{Author}
\strong{Maintainer}: Daniel Heimgartner \email{d.heimgartners@gmail.com} (\Rhref{https://orcid.org/0000-0002-0643-8690}{ORCID}) [copyright holder]

\end{Author}

%%%%%

\inputencoding{utf8}
\HeaderA{kfplot}{Comparison Plot Method for List of OPSR kfold Objects}{kfplot}
%
\begin{Description}
Comparison Plot Method for List of OPSR kfold Objects
\end{Description}
%
\begin{Usage}
\begin{verbatim}
kfplot(
  x,
  i = c("ll_mean", "ll_p_mean", "r2"),
  main = NULL,
  xlab = NULL,
  ylab = NULL,
  ...
)
\end{verbatim}
\end{Usage}
%
\begin{Arguments}
\begin{ldescription}
\item[\code{x}] a list of objects of class \code{"opsr.kfold"}.

\item[\code{i}] the loss to extract (see 'Value' in \code{\LinkA{loss}{loss}}). One of \code{"ll\_mean"},
\code{"ll\_p\_mean"} or \code{"r2"}.

\item[\code{main}] a main title for the plot, see also \code{title}.

\item[\code{xlab}] a label for the x axis.

\item[\code{ylab}] a label for the y axis.

\item[\code{...}] further arguments passed to \code{boxplot}.
\end{ldescription}
\end{Arguments}
%
\begin{Value}
See 'Value' in \code{boxplot}.
\end{Value}
%
\begin{SeeAlso}
\code{\LinkA{loss}{loss}}
\end{SeeAlso}

%%%%%

\inputencoding{utf8}
\HeaderA{list2df}{List to Data Frame}{list2df}
%
\begin{Description}
Converts a list of vectors to a data frame.
\end{Description}
%
\begin{Usage}
\begin{verbatim}
list2df(x)
\end{verbatim}
\end{Usage}
%
\begin{Arguments}
\begin{ldescription}
\item[\code{x}] a list of vectors.
\end{ldescription}
\end{Arguments}
%
\begin{Value}
A data frame.
\end{Value}

%%%%%

\inputencoding{utf8}
\HeaderA{loss}{Out-Of-Sample Goodness of Fit Indicators}{loss}
%
\begin{Description}
Computes a loss for out-of-sample data points.
\end{Description}
%
\begin{Usage}
\begin{verbatim}
loss(object, data, test_ind, ...)
\end{verbatim}
\end{Usage}
%
\begin{Arguments}
\begin{ldescription}
\item[\code{object}] an object of class \code{"opsr"}.

\item[\code{data}] a data frame containing the variables in the model (usually from
a call to \code{\LinkA{model.frame.opsr}{model.frame.opsr}}).

\item[\code{test\_ind}] indicator for the test data.

\item[\code{...}] addditional arguments passed to \code{\LinkA{opsr\_from\_opsr}{opsr.Rul.from.Rul.opsr}}.
\end{ldescription}
\end{Arguments}
%
\begin{Value}
A list containing the losses:
\begin{itemize}

\item{} \code{z}: The regime/treatment membership of the test data.
\item{} \code{ll}: The out-of-sample log-likelihood of the test data.
\item{} \code{ll\_mean}: \code{ll} averaged over each regime and in total.
\item{} \code{ll\_p}: The out-of-sample log-likelihood of the test data for the selection
process.
\item{} \code{ll\_p\_mean}: \code{ll\_p} averaged over each regime and in total.
\item{} \code{r2}: Coefficient of determination for each regime and in total.

\end{itemize}

\end{Value}

%%%%%

\inputencoding{utf8}
\HeaderA{opsr\_ate}{Treatment Effect Computations for OPSR Model Fits}{opsr.Rul.ate}
%
\begin{Description}
Treatment Effect Computations for OPSR Model Fits
\end{Description}
%
\begin{Usage}
\begin{verbatim}
opsr_ate(object, type, weights = NULL, ...)
\end{verbatim}
\end{Usage}
%
\begin{Arguments}
\begin{ldescription}
\item[\code{object}] object an object of class \code{"opsr"}.

\item[\code{type}] see \code{\LinkA{predict.opsr}{predict.opsr}} for details.

\item[\code{weights}] a vector of weights. If \code{NULL} then weights from \code{object} will
be used.

\item[\code{...}] additional arguments passed to \code{\LinkA{predict.opsr}{predict.opsr}}.
\end{ldescription}
\end{Arguments}
%
\begin{Details}
This function only prepares the input to a further call to \code{\LinkA{summary.opsr.ate}{summary.opsr.ate}}.
\end{Details}
%
\begin{Value}
An object of class \code{"opsr.ate"}.
\end{Value}
%
\begin{SeeAlso}
\code{\LinkA{summary.opsr.ate}{summary.opsr.ate}}
\end{SeeAlso}
%
\begin{Examples}
\begin{ExampleCode}
sim_dat <- OPSR::opsr_simulate()
dat <- sim_dat$data
weights <- runif(nrow(dat))
fit <- OPSR::opsr(ys | yo ~ xs1 + xs2 | xo1 + xo2, dat = dat,
                  weights = weights, printLevel = 0)
ate <- opsr_ate(fit, type = "response")
print(ate)
summary(ate)

ate_w <- opsr_ate(fit, type = "response", weights = rep(1, nrow(dat)))
summary(ate_w)

pairs(ate)
\end{ExampleCode}
\end{Examples}

%%%%%

\inputencoding{utf8}
\HeaderA{opsr\_from\_opsr}{Refit an OPSR Model on New Data}{opsr.Rul.from.Rul.opsr}
%
\begin{Description}
Refit an OPSR Model on New Data
\end{Description}
%
\begin{Usage}
\begin{verbatim}
opsr_from_opsr(object, data, ...)
\end{verbatim}
\end{Usage}
%
\begin{Arguments}
\begin{ldescription}
\item[\code{object}] an object of class \code{"opsr"}.

\item[\code{data}] a data frame containing the variables in the model.

\item[\code{...}] additional arguments passed to \code{\LinkA{opsr}{opsr}}.
\end{ldescription}
\end{Arguments}
%
\begin{Value}
An object of class \code{"opsr"}
\end{Value}

%%%%%

\inputencoding{utf8}
\HeaderA{opsr\_kfold}{k-Fold Cross-Validation for OPSR Model Fits}{opsr.Rul.kfold}
%
\begin{Description}
Computes out-of-sample losses.
\end{Description}
%
\begin{Usage}
\begin{verbatim}
opsr_kfold(object, k = 10, verbose = TRUE, ...)
\end{verbatim}
\end{Usage}
%
\begin{Arguments}
\begin{ldescription}
\item[\code{object}] an object of class \code{"opsr"}.

\item[\code{k}] number of folds.

\item[\code{verbose}] if \code{TRUE}, prints working information during iterations over
folds.

\item[\code{...}] additional arguments passed to \code{\LinkA{opsr\_from\_opsr}{opsr.Rul.from.Rul.opsr}}
\end{ldescription}
\end{Arguments}
%
\begin{Details}
The returned object is of length \code{k}, where each element contains the losses
as computed by \code{\LinkA{loss}{loss}}.
\end{Details}
%
\begin{Value}
An object of class \code{"opsr.kfold"}. See 'Details' section for more
information.
\end{Value}
%
\begin{SeeAlso}
\code{\LinkA{opsr\_from\_opsr}{opsr.Rul.from.Rul.opsr}}, \code{\LinkA{loss}{loss}}, \code{\LinkA{[.opsr.kfold}{[.opsr.kfold}}
\end{SeeAlso}
%
\begin{Examples}
\begin{ExampleCode}
sim_dat <- OPSR::opsr_simulate()
dat <- sim_dat$data
dat$xo3 <- runif(n = nrow(dat))
dat$xo4 <- factor(sample(c("level1", "leve2", "level3"), nrow(dat),
                  replace = TRUE))
f <- ys | yo ~ xs1 + xs2 + log(xo3) | xo1 + xo2 + xo3 + xo4 |
  xo1 + xo2 + xo3 | xo1 + xo2
fit <- OPSR::opsr(f, dat, printLevel = 0)
fit_select <- opsr_select(fit, loss = "bic", printLevel = 0)
kfold <- opsr_kfold(fit, printLevel = 0)
kfold_select <- opsr_kfold(fit_select, printLevel = 0)

## extract
ll_mean <- kfold["ll_mean"]
list2df(ll_mean)

## plot
colvec <- c("#ff8811","#48a9a6")
kfplot(list(kfold, kfold_select), col = colvec)
legend("bottomleft", legend = c("fit", "fit_select"), fill = colvec,
       title = "Model", title.font = 2, bty = "n")
\end{ExampleCode}
\end{Examples}

%%%%%

\inputencoding{utf8}
\HeaderA{opsr\_select}{Select a Model in a Stepwise Algorithm}{opsr.Rul.select}
%
\begin{Description}
Iteratively calls \code{\LinkA{opsr\_step}{opsr.Rul.step}} and compares the resulting model to the
current winner (as evaluated by a selected loss function).
\end{Description}
%
\begin{Usage}
\begin{verbatim}
opsr_select(
  object,
  pseq = seq(0.9, 0.1, by = -0.1),
  log = new.env(),
  verbose = TRUE,
  loss = c("aic", "bic", "lrt"),
  ...
)
\end{verbatim}
\end{Usage}
%
\begin{Arguments}
\begin{ldescription}
\item[\code{object}] an object of class \code{"opsr"}.

\item[\code{pseq}] sequence of p-value thresholds (each of which is passed as \code{pval}
to \code{\LinkA{opsr\_step}{opsr.Rul.step}}).

\item[\code{log}] environment to keep track of changes to \code{object} (in particular
variables being eliminated).

\item[\code{verbose}] if \code{TRUE}, prints working information during computation.

\item[\code{loss}] The loss function for model comparison. Can be abbreviated.
See 'Details' section for more information.

\item[\code{...}] additional arguments passed to \code{\LinkA{opsr\_select}{opsr.Rul.select}}
\end{ldescription}
\end{Arguments}
%
\begin{Details}
Currently three loss functions are available which can be selected via the
\code{loss} argument. The loss is then computed for the two models to be compared
and a winner is selected. Can be one of \code{"aic"} for AIC, \code{"bic"} for BIC and
\code{"lrt"} for a likelihood ratio test.
\end{Details}
%
\begin{Value}
An object of class \code{"opsr.select"}.
\end{Value}
%
\begin{SeeAlso}
\code{\LinkA{opsr\_select}{opsr.Rul.select}}
\end{SeeAlso}
%
\begin{Examples}
\begin{ExampleCode}
sim_dat <- OPSR::opsr_simulate()
dat <- sim_dat$data
dat$xo3 <- runif(n = nrow(dat))
dat$xo4 <- factor(sample(c("level1", "leve2", "level3"), nrow(dat),
                  replace = TRUE))
f <- ys | yo ~ xs1 + xs2 + log(xo3) | xo1 + xo2 + xo3 + xo4 |
  xo1 + xo2 + xo3 | xo1 + xo2
fit <- OPSR::opsr(f, dat, printLevel = 0)
fit_select <- opsr_select(fit, loss = "aic", printLevel = 0)
print(fit_select)
\end{ExampleCode}
\end{Examples}

%%%%%

\inputencoding{utf8}
\HeaderA{opsr\_step}{Step Function for OPSR Model Fits}{opsr.Rul.step}
%
\begin{Description}
Excludes all coefficients with p-values below \code{pval} and fits again.
\end{Description}
%
\begin{Usage}
\begin{verbatim}
opsr_step(object, pval, log = new.env(), .step = 1, ...)
\end{verbatim}
\end{Usage}
%
\begin{Arguments}
\begin{ldescription}
\item[\code{object}] an object of class \code{"opsr"}.

\item[\code{pval}] coefficients with p-values < \code{pval} are dropped.

\item[\code{log}] environment to keep track of changes to \code{object} (in particular
variables being eliminated).

\item[\code{.step}] used to generate identifier in \code{log} environment. Used in
\code{\LinkA{opsr\_select}{opsr.Rul.select}}.

\item[\code{...}] additional arguments passed to \code{update} (and hence \code{\LinkA{opsr}{opsr}}).
\end{ldescription}
\end{Arguments}
%
\begin{Value}
An object of class \code{"opsr"}.
\end{Value}
%
\begin{SeeAlso}
\code{\LinkA{opsr\_select}{opsr.Rul.select}}
\end{SeeAlso}
%
\begin{Examples}
\begin{ExampleCode}
sim_dat <- OPSR::opsr_simulate()
dat <- sim_dat$data
dat$xo3 <- runif(n = nrow(dat))
dat$xo4 <- factor(sample(c("level1", "leve2", "level3"), nrow(dat),
                  replace = TRUE))
f <- ys | yo ~ xs1 + xs2 + log(xo3) | xo1 + xo2 + xo3 + xo4 |
  xo1 + xo2 + xo3 | xo1 + xo2
fit <- OPSR::opsr(f, dat, printLevel = 0)
fit_step <- opsr_step(fit, pval = 0.1)
texreg::screenreg(list(fit, fit_step))
\end{ExampleCode}
\end{Examples}

%%%%%

\inputencoding{utf8}
\HeaderA{pairs.opsr.ate}{Pairs Plot for OPSR ATE Objects}{pairs.opsr.ate}
%
\begin{Description}
Pairs Plot for OPSR ATE Objects
\end{Description}
%
\begin{Usage}
\begin{verbatim}
## S3 method for class 'opsr.ate'
pairs(
  x,
  pch = 21,
  labels.diag = paste0("T", 1:x$nReg),
  labels.reg = paste0("G", 1:x$nReg),
  col = 1:x$nReg,
  add.rug = TRUE,
  lower.digits = 0,
  diag.digits = 0,
  lwd.dens = 1.5,
  diag.cex.text = 1,
  upper.digits = 2,
  upper.cex.text = 2,
  prefix = "",
  postfix = "",
  lty.diag = 1,
  ...
)
\end{verbatim}
\end{Usage}
%
\begin{Arguments}
\begin{ldescription}
\item[\code{x}] an object of class \code{"opsr.ate"}.

\item[\code{pch}] plotting 'character', i.e., symbol to use. See also \code{pch}.

\item[\code{labels.diag}] labels used in the diagonal panels.

\item[\code{labels.reg}] labels for the treatment regimes.

\item[\code{col}] colour vector.

\item[\code{add.rug}] if \code{TRUE}, adds rugs to the lower panels.

\item[\code{lower.digits}] rounding of the digits in the lower panel.

\item[\code{diag.digits}] rounding of the digits in the diagonal panel.

\item[\code{lwd.dens}] linewidth of the densities in the diagonal panel.

\item[\code{diag.cex.text}] \code{cex} for the text in the diagonal panel.

\item[\code{upper.digits}] rounding of the digits in the upper panel.

\item[\code{upper.cex.text}] \code{cex} for the text in the upper panel.

\item[\code{prefix}] for the number plotted in the upper panel.

\item[\code{postfix}] for the number plotted in the upper panel.

\item[\code{lty.diag}] linetype for the diagonal panel.

\item[\code{...}] further arguments passed to or from other methods.
\end{ldescription}
\end{Arguments}
%
\begin{Details}
Presents all potential counterfactual outcomes. The diagonal depicts
distributions in any given treatment regime and separate by the current
(factual) treatment group. The weighted mean values are shown as red numbers.
The lower triangular panels compare the model-implied (predicted) outcomes
of two treatment regimes again separate by current treatment group. The red
line indicates the 45-degree line of equal outcomes while the red squares
depict again the weighted mean values. The upper triangular panels show
(weighted) average treatment effects.
\end{Details}
%
\begin{Value}
Returns \code{x} invisibly.
\end{Value}
%
\begin{SeeAlso}
\code{pairs}
\end{SeeAlso}
%
\begin{Examples}
\begin{ExampleCode}
sim_dat <- OPSR::opsr_simulate()
dat <- sim_dat$data
weights <- runif(nrow(dat))
fit <- OPSR::opsr(ys | yo ~ xs1 + xs2 | xo1 + xo2, dat = dat,
                  weights = weights, printLevel = 0)
ate <- opsr_ate(fit, type = "response")
print(ate)
summary(ate)

ate_w <- opsr_ate(fit, type = "response", weights = rep(1, nrow(dat)))
summary(ate_w)

pairs(ate)
\end{ExampleCode}
\end{Examples}

%%%%%

\inputencoding{utf8}
\HeaderA{plot.opsr.kfold}{Plot Method for OPSR kfold Objects}{plot.opsr.kfold}
%
\begin{Description}
Plot Method for OPSR kfold Objects
\end{Description}
%
\begin{Usage}
\begin{verbatim}
## S3 method for class 'opsr.kfold'
plot(
  x,
  i = c("ll_mean", "ll_p_mean", "r2"),
  main = NULL,
  xlab = NULL,
  ylab = NULL,
  ...
)
\end{verbatim}
\end{Usage}
%
\begin{Arguments}
\begin{ldescription}
\item[\code{x}] an object of class \code{"opsr.kfold"}.

\item[\code{i}] the loss to extract (see 'Value' in \code{\LinkA{loss}{loss}}). One of \code{"ll\_mean"},
\code{"ll\_p\_mean"} or \code{"r2"}.

\item[\code{main}] a main title for the plot, see also \code{title}.

\item[\code{xlab}] a label for the x axis.

\item[\code{ylab}] a label for the y axis.

\item[\code{...}] further arguments passed to \code{boxplot}.
\end{ldescription}
\end{Arguments}
%
\begin{Value}
See 'Value' in \code{boxplot}.
\end{Value}
%
\begin{SeeAlso}
\code{\LinkA{loss}{loss}}
\end{SeeAlso}

%%%%%

\inputencoding{utf8}
\HeaderA{print.ate}{Print Method for ATE Objects}{print.ate}
%
\begin{Description}
Print Method for ATE Objects
\end{Description}
%
\begin{Usage}
\begin{verbatim}
## S3 method for class 'ate'
print(x, digits = max(3L, getOption("digits") - 3L),
      signif.legend = TRUE, ...)
\end{verbatim}
\end{Usage}
%
\begin{Arguments}
\begin{ldescription}
\item[\code{x}] an object of class \code{"ate"}.

\item[\code{digits}] minimum number of significant digits to be used for most numbers
(passed to \code{stats::printCoefmat}).

\item[\code{signif.legend}] if \code{TRUE}, a legend for the 'significance stars' is printed.

\item[\code{...}] further arguments passed to or from other methods.
\end{ldescription}
\end{Arguments}
%
\begin{Value}
Prints \code{x} in 'pretty' form and returns reformatted treatment effects
invisibly.
\end{Value}

%%%%%

\inputencoding{utf8}
\HeaderA{print.opsr.ate}{Print Method for OPSR ATE Objects}{print.opsr.ate}
%
\begin{Description}
Print Method for OPSR ATE Objects
\end{Description}
%
\begin{Usage}
\begin{verbatim}
## S3 method for class 'opsr.ate'
print(x, ...)
\end{verbatim}
\end{Usage}
%
\begin{Arguments}
\begin{ldescription}
\item[\code{x}] an object of class \code{"opsr.ate"}.

\item[\code{...}] further arguments passed to \code{\LinkA{summary.opsr.ate}{summary.opsr.ate}}.
\end{ldescription}
\end{Arguments}
%
\begin{Details}
This is just a wrapper around \code{\LinkA{summary.opsr.ate}{summary.opsr.ate}} and a subsequent call to
\code{\LinkA{print.summary.opsr.ate}{print.summary.opsr.ate}}.
\end{Details}
%
\begin{Value}
Returns \code{x} invisibly.
\end{Value}
%
\begin{SeeAlso}
\code{\LinkA{print.summary.opsr.ate}{print.summary.opsr.ate}}
\end{SeeAlso}

%%%%%

\inputencoding{utf8}
\HeaderA{print.opsr.select}{Print Method for OPSR Select Objects}{print.opsr.select}
%
\begin{Description}
Prints the original model and the final model (winner) as well as the
stepwise model path.
\end{Description}
%
\begin{Usage}
\begin{verbatim}
## S3 method for class 'opsr.select'
print(
  x,
  digits = max(3L, getOption("digits") - 3L),
  print.call = TRUE,
  print.elim.hist = TRUE,
  ...
)
\end{verbatim}
\end{Usage}
%
\begin{Arguments}
\begin{ldescription}
\item[\code{x}] an object of class \code{"opsr.select"}.

\item[\code{digits}] minimum number of significant digits to be used for most numbers (passed to \code{stats::printCoefmat}).

\item[\code{print.call}] if \code{TRUE}, prints the underlying \code{\LinkA{opsr\_select}{opsr.Rul.select}} call.

\item[\code{print.elim.hist}] if \code{TRUE}, prints the elimination history. See
'Details' section for more information.

\item[\code{...}] further arguments passed to or from other methods.
\end{ldescription}
\end{Arguments}
%
\begin{Value}
Prints \code{x} in 'pretty' form and returns it invisibly.
\end{Value}

%%%%%

\inputencoding{utf8}
\HeaderA{print.summary.opsr.ate}{Print Method for Summary OPSR ATE Objects}{print.summary.opsr.ate}
%
\begin{Description}
Print Method for Summary OPSR ATE Objects
\end{Description}
%
\begin{Usage}
\begin{verbatim}
## S3 method for class 'summary.opsr.ate'
print(x, digits = max(3L, getOption("digits") - 3L),
      print.call = FALSE, ...)
\end{verbatim}
\end{Usage}
%
\begin{Arguments}
\begin{ldescription}
\item[\code{x}] an object of class \code{"summary.opsr.ate"}.

\item[\code{digits}] minimum number of significant digits to be used for most numbers (passed to \code{stats::printCoefmat}).

\item[\code{print.call}] if \code{TRUE}, prints the underlying call.

\item[\code{...}] further arguments passed to or from other methods.
\end{ldescription}
\end{Arguments}
%
\begin{Value}
Prints \code{x} in 'pretty' form and returns it invisibly.
\end{Value}

%%%%%

\inputencoding{utf8}
\HeaderA{print.te}{Print Method for TE Objects}{print.te}
%
\begin{Description}
Print Method for TE Objects
\end{Description}
%
\begin{Usage}
\begin{verbatim}
## S3 method for class 'te'
print(x, digits = max(3L, getOption("digits") - 3L),
      signif.legend = TRUE, ...)
\end{verbatim}
\end{Usage}
%
\begin{Arguments}
\begin{ldescription}
\item[\code{x}] an object of class \code{"te"}.

\item[\code{digits}] minimum number of significant digits to be used for most numbers
(passed to \code{stats::printCoefmat}).

\item[\code{signif.legend}] if \code{TRUE}, a legend for the 'significance stars' is printed.

\item[\code{...}] further arguments passed to or from other methods.
\end{ldescription}
\end{Arguments}
%
\begin{Value}
Prints \code{x} in 'pretty' form and returns reformatted treatment effects
invisibly.
\end{Value}

%%%%%

\inputencoding{utf8}
\HeaderA{[.opsr.kfold}{Extract Method for OPSR kfold Object}{[.opsr.kfold}
%
\begin{Description}
Extract Method for OPSR kfold Object
\end{Description}
%
\begin{Usage}
\begin{verbatim}
## S3 method for class 'opsr.kfold'
x[i]
\end{verbatim}
\end{Usage}
%
\begin{Arguments}
\begin{ldescription}
\item[\code{x}] an object of class \code{"opsr.kfold"}.

\item[\code{i}] the loss to extract (see 'Value' in \code{\LinkA{loss}{loss}}).
\end{ldescription}
\end{Arguments}
%
\begin{Value}
A list of length k (where k is the number of folds).
\end{Value}
%
\begin{SeeAlso}
\code{\LinkA{loss}{loss}}
\end{SeeAlso}

%%%%%

\inputencoding{utf8}
\HeaderA{summary.opsr.ate}{Summarizing OPSR ATE Objects}{summary.opsr.ate}
%
\begin{Description}
This function computes weighted treatment effects and corresponding weighted
paired t-tests.
\end{Description}
%
\begin{Usage}
\begin{verbatim}
## S3 method for class 'opsr.ate'
summary(object, ...)
\end{verbatim}
\end{Usage}
%
\begin{Arguments}
\begin{ldescription}
\item[\code{object}] an object of class \code{"opsr.ate"}.

\item[\code{...}] further arguments passed to or from other methods.
\end{ldescription}
\end{Arguments}
%
\begin{Value}
An object of class \code{"summary.opsr.ate"} containing among others:
\begin{itemize}

\item{} \code{ate}: An object of class \code{"ate"}. See also \code{\LinkA{print.ate}{print.ate}}.
\item{} \code{te}: An object of class \code{"te"}. See also \code{\LinkA{print.te}{print.te}}.

\end{itemize}


The p-values of the weighted paired t-test are attached as attributes.
\end{Value}

%%%%%

\inputencoding{utf8}
\HeaderA{timeuse}{TimeUse+ Data}{timeuse}
\keyword{datasets}{timeuse}
%
\begin{Description}
Travel diary data from the TimeUse+ study \citep{Winkler+Meister+Axhausen:2024}.
\end{Description}
%
\begin{Usage}
\begin{verbatim}
timeuse
\end{verbatim}
\end{Usage}
%
\begin{Format}
Data frame with numeric and factor columns
\begin{description}

\item[uid] Respondent ID
\item[weight] Sample weight
\item[wfh] Telework status
\item[wfh\_days] Telework frequency (days/week)
\item[weekly\_km] Weekly kilometers traveled
\item[log\_weekly\_km] Log weekly kilometers traveled
\item[weekly\_n] Weekly number of trips
\item[log\_weekly\_n] Log weekly number of trips
\item[start\_tracking] Month when tracking started
\item[commute] One-way commute distance (km)
\item[log\_commute\_km] Log one-way commute distance
\item[age] Age
\item[car\_access] Car access
\item[dogs] Has dogs
\item[driverlicense] Has driverlicense
\item[educ\_higher] Higher education
\item[fixed\_workplace] Has fixed workplace
\item[grocery\_shopper] Main responsible for grocery shopping
\item[hh\_income] Household income
\item[hh\_size] Household size
\item[isco\_clerical] ISCO clerical
\item[isco\_craft] ISCO craft
\item[isco\_elementary] ISCO elementary
\item[isco\_managers] ISCO managers
\item[isco\_plant] ISCO plant and machine
\item[isco\_professionals] ISCO professionals
\item[isco\_service] ISCO service and sales
\item[isco\_agri] ISCO agriculture
\item[isco\_tech] ISCO technicians
\item[married] Married
\item[n\_children] Number of children
\item[freq\_onl\_order] Frequent online shopper
\item[parking\_home] Parking available at home
\item[parking\_work] Parking available at work
\item[permanent\_employed] Permanent employment
\item[rents\_home] Tenant
\item[res\_loc] Residential location
\item[sex\_male] Male
\item[shift\_work] Works in shifts
\item[swiss] Nationality Swiss
\item[vacation] Vacation during study
\item[workload] Workload (in \% of full-time equivalent)
\item[young\_kids] Has young kids
\item[wfh\_allowed] Employer allows telework
\item[teleworkability] Number of days worked from home during COVID-related lockdowns

\end{description}

\end{Format}
%
\begin{References}
\bibentry{Winkler+Meister+Axhausen:2024}
\end{References}

