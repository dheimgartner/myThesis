%%%%%

\inputencoding{utf8}
\HeaderA{OPSR-package}{OPSR: Ordered Probit Switching Regression}{OPSR.Rdash.package}
\aliasA{OPSR}{OPSR-package}{OPSR}
%
\begin{Description}
Estimates ordered probit switching regression models - a Heckman type selection model with an ordinal selection and continuous outcomes. Different model specifications are allowed for each treatment/regime. For more details on the method, see \citet{Wang+Mokhtarian:2024} or \citet{Chiburis+Lokshin:2007}.
\end{Description}
%
\begin{Author}
\strong{Maintainer}: Daniel Heimgartner \email{d.heimgartners@gmail.com} (\Rhref{https://orcid.org/0000-0002-0643-8690}{ORCID}) [copyright holder]

Authors:
\begin{itemize}

\item{} Xinyi Wang \email{xinyi174@mit.edu} (\Rhref{https://orcid.org/0000-0002-3564-9147}{ORCID})

\end{itemize}


\end{Author}
%
\begin{SeeAlso}
Useful links:
\begin{itemize}

\item{} \url{https://github.com/dheimgartner/OPSR}
\item{} Report bugs at \url{https://github.com/dheimgartner/OPSR/issues}

\end{itemize}


\end{SeeAlso}

%%%%%

\inputencoding{utf8}
\HeaderA{anova.opsr}{ANOVA for OPSR Model Fits}{anova.opsr}
%
\begin{Description}
Conducts likelihood ratio tests for one or more OPSR model fits.
\end{Description}
%
\begin{Usage}
\begin{verbatim}
## S3 method for class 'opsr'
anova(object, ...)
\end{verbatim}
\end{Usage}
%
\begin{Arguments}
\begin{ldescription}
\item[\code{object}] an object of class \code{"opsr"}.

\item[\code{...}] additional objects of class \code{"opsr"}. See also the 'Details' section.
\end{ldescription}
\end{Arguments}
%
\begin{Details}
If only a single object is passed then the model is compared to the null model
(\code{\LinkA{opsr\_null\_model}{opsr.Rul.null.Rul.model}}). If more than one object is specified, a likelihood ratio
test is conducted for each pair of neighboring models. It is conventional to
list the models from smallest to largest, but this is up to the user.
\end{Details}
%
\begin{Value}
An object of class \code{"anova.opsr"}.
\end{Value}
%
\begin{SeeAlso}
\code{stats::anova}, \code{\LinkA{print.anova.opsr}{print.anova.opsr}}
\end{SeeAlso}
%
\begin{Examples}
\begin{ExampleCode}
sim_dat <- opsr_simulate()
dat <- sim_dat$data
model <- ys | yo ~ xs1 + xs2 | xo1 + xo2
fit <- opsr(model, dat)
fit_null <- opsr_null_model(fit)
fit_intercept <- update(fit, ~ . | 1)

anova(fit)
anova(fit_null, fit_intercept, fit)
\end{ExampleCode}
\end{Examples}

%%%%%

\inputencoding{utf8}
\HeaderA{extract,opsr-method}{Extract Method for OPSR Model Fits}{extract,opsr.Rdash.method}
\aliasA{extract.opsr}{extract,opsr-method}{extract.opsr}
%
\begin{Description}
This is the main method called when using functions from the \code{texreg-package}.
\end{Description}
%
\begin{Usage}
\begin{verbatim}
## S4 method for signature 'opsr'
extract(
  model,
  beside = FALSE,
  include.structural = TRUE,
  include.selection = TRUE,
  include.outcome = TRUE,
  include.pseudoR2 = FALSE,
  include.R2 = FALSE,
  repeat.gofs = FALSE,
  ...
)
\end{verbatim}
\end{Usage}
%
\begin{Arguments}
\begin{ldescription}
\item[\code{model}] an object of class \code{"opsr"}.

\item[\code{beside}] if \code{TRUE}, prints structural, selection and outcome coefficients side-by-side.

\item[\code{include.structural}] whether or not structural coefficients should be printed.

\item[\code{include.selection}] whether or not selection coefficients should be printed.

\item[\code{include.outcome}] whether or not outcome coefficients should be printed.

\item[\code{include.pseudoR2}] whether or not the pseudo R2 statistic for the selection
component should be printed. See also the 'Details' section.

\item[\code{include.R2}] whether or not the R2 statistic for the outcome components
should be printed.

\item[\code{repeat.gofs}] if \code{beside = TRUE} whether or not to repeat the gofs.

\item[\code{...}] additional arguments passed to \code{\LinkA{summary.opsr}{summary.opsr}}.
\end{ldescription}
\end{Arguments}
%
\begin{Details}
The \code{extract} method is called internally. Higher-level functions from the
\code{texreg-package} pass arguments via \code{...} to \code{extract}.

\code{include.pseudoR2} reports both the "equally likely" (EL) and "market share" (MS)
pseudo R2.
\end{Details}
%
\begin{Value}
A \code{texreg-class} object representing the statistical model.
\end{Value}
%
\begin{SeeAlso}
\code{texreg-package}, \code{texreg::texreg}, \code{texreg::screenreg} and related functions.
\end{SeeAlso}
%
\begin{Examples}
\begin{ExampleCode}
sim_dat <- opsr_simulate()
dat <- sim_dat$data
model <- ys | yo ~ xs1 + xs2 | xo1 + xo2
fit <- opsr(model, dat)
fit_null <- opsr_null_model(fit)
fit_intercept <- update(fit, ~ . | 1)

texreg::screenreg(fit)
texreg::screenreg(fit, beside = TRUE)
texreg::screenreg(fit, beside = TRUE, include.pseudoR2 = TRUE,
                  include.R2 = TRUE)
texreg::screenreg(list(fit_null, fit_intercept, fit))
\end{ExampleCode}
\end{Examples}

%%%%%

\inputencoding{utf8}
\HeaderA{loglik\_cpp}{Interface to C++ Log-Likelihood Implementation}{loglik.Rul.cpp}
%
\begin{Description}
This is the main computation engine wrapped by \code{\LinkA{opsr.fit}{opsr.fit}}.
\end{Description}
%
\begin{Usage}
\begin{verbatim}
loglik_cpp(theta, W, X, Y, weights, nReg, nThreads)
\end{verbatim}
\end{Usage}
%
\begin{Arguments}
\begin{ldescription}
\item[\code{theta}] named coefficient vector as parsed from formula interface \code{\LinkA{opsr}{opsr}}.

\item[\code{W}] list of matrices with explanatory variables for selection process for each regime.

\item[\code{X}] list of matrices with expalanatory varialbes for outcome process for each regime.

\item[\code{Y}] list of vectors with continuous outcomes for each regime.

\item[\code{weights}] vector of weights. See also \code{\LinkA{opsr}{opsr}}.

\item[\code{nReg}] integer number of regimes.

\item[\code{nThreads}] number of threads to be used by \code{OpenMP} (should be max. \code{nReg}).
\end{ldescription}
\end{Arguments}
%
\begin{Value}
Numeric vector of (weighted) log-likelihood contributions.
\end{Value}
%
\begin{SeeAlso}
\code{\LinkA{opsr.fit}{opsr.fit}}, \code{\LinkA{loglik\_R}{loglik.Rul.R}}
\end{SeeAlso}

%%%%%

\inputencoding{utf8}
\HeaderA{loglik\_R}{R-based Log-Likelihood Implementation}{loglik.Rul.R}
\keyword{internal}{loglik\_R}
%
\begin{Description}
R-based Log-Likelihood Implementation
\end{Description}
%
\begin{Usage}
\begin{verbatim}
loglik_R(theta, W, X, Y, weights, nReg, ...)
\end{verbatim}
\end{Usage}
%
\begin{SeeAlso}
\code{\LinkA{loglik\_cpp}{loglik.Rul.cpp}}, \code{\LinkA{opsr.fit}{opsr.fit}}
\end{SeeAlso}

%%%%%

\inputencoding{utf8}
\HeaderA{model.frame.opsr}{Extracting the Model Frame from OPSR Model Fits}{model.frame.opsr}
%
\begin{Description}
Extracting the Model Frame from OPSR Model Fits
\end{Description}
%
\begin{Usage}
\begin{verbatim}
## S3 method for class 'opsr'
model.frame(formula, ...)
\end{verbatim}
\end{Usage}
%
\begin{Arguments}
\begin{ldescription}
\item[\code{formula}] an object of class \code{"opsr"}.

\item[\code{...}] a mix of further arguments such as \code{data}, \code{na.action} or \code{subset},
passed to the default method.
\end{ldescription}
\end{Arguments}
%
\begin{Value}
A \code{data.frame} containing the variables used in \code{formula\$formula}.
\end{Value}
%
\begin{SeeAlso}
\code{stats::model.frame}
\end{SeeAlso}

%%%%%

\inputencoding{utf8}
\HeaderA{model.matrix.opsr}{Construct Design Matrices for OPSR Model Fits}{model.matrix.opsr}
%
\begin{Description}
Construct Design Matrices for OPSR Model Fits
\end{Description}
%
\begin{Usage}
\begin{verbatim}
## S3 method for class 'opsr'
model.matrix(object, data, .filter = NULL, ...)
\end{verbatim}
\end{Usage}
%
\begin{Arguments}
\begin{ldescription}
\item[\code{object}] an object of class \code{"opsr"}.

\item[\code{data}] a data frame containing the terms from \code{object\$formula}. Passed to
\code{\LinkA{model.frame.opsr}{model.frame.opsr}}. Can be omitted.

\item[\code{.filter}] used internally in \code{\LinkA{predict.opsr}{predict.opsr}} for counterfactual predictions.

\item[\code{...}] further arguments passed to or from other methods.
\end{ldescription}
\end{Arguments}
%
\begin{Value}
A list of lists with the design matrices \code{W} (selection process) and
\code{X} (outcome process). Both of these lists have \code{object\$nReg} elements (a
separate design matrix for each regime).
\end{Value}
%
\begin{SeeAlso}
\code{\LinkA{model.frame.opsr}{model.frame.opsr}}, \code{stats::model.matrix}
\end{SeeAlso}

%%%%%

\inputencoding{utf8}
\HeaderA{opsr\_2step}{Heckman Two-Step Estimation}{opsr.Rul.2step}
%
\begin{Description}
This is a utility function, used in \code{\LinkA{opsr}{opsr}} and should not be used directly.
Tow-step estimation procedure to generate reasonable starting values.
\end{Description}
%
\begin{Usage}
\begin{verbatim}
opsr_2step(W, Xs, Z, Ys)
\end{verbatim}
\end{Usage}
%
\begin{Arguments}
\begin{ldescription}
\item[\code{W}] matrix with explanatory variables for selection process.

\item[\code{Xs}] list of matrices with expalanatory varialbes for outcome process for each regime.

\item[\code{Z}] vector with ordinal outcomes (in integer increasing fashion).

\item[\code{Ys}] list of vectors with continuous outcomes for each regime.
\end{ldescription}
\end{Arguments}
%
\begin{Details}
These estimates can be retrieved by specifying \code{.get2step = TRUE} in \code{\LinkA{opsr}{opsr}}.
\end{Details}
%
\begin{Value}
Named vector with starting values passed to \code{\LinkA{opsr.fit}{opsr.fit}}.
\end{Value}
%
\begin{Section}{Remark}

Since the Heckman two-step estimator includes an estimate in the second step
regression, the resulting OLS standard errors and heteroskedasticity-robust
standard errors are incorrect \citep{Greene:2002}.
\end{Section}
%
\begin{References}
\bibentry{Greene:2002}
\end{References}
%
\begin{SeeAlso}
\code{\LinkA{opsr.fit}{opsr.fit}}, \code{\LinkA{opsr\_prepare\_coefs}{opsr.Rul.prepare.Rul.coefs}}
\end{SeeAlso}

%%%%%

\inputencoding{utf8}
\HeaderA{opsr\_check\_omp}{Check Whether \code{OpenMP} is Available}{opsr.Rul.check.Rul.omp}
%
\begin{Description}
Check Whether \code{OpenMP} is Available
\end{Description}
%
\begin{Usage}
\begin{verbatim}
opsr_check_omp()
\end{verbatim}
\end{Usage}
%
\begin{Value}
boolean
\end{Value}

%%%%%

\inputencoding{utf8}
\HeaderA{opsr\_check\_start}{Check the User-Specified Starting Values}{opsr.Rul.check.Rul.start}
%
\begin{Description}
This is a utility function, used in \code{\LinkA{opsr}{opsr}} and should not be used directly.
It is included here to document the expected structure of \code{\LinkA{opsr}{opsr}}'s \code{start} argument.
Makes sure, the start vector conforms to the expected structure. Adds the
expected parameter names to the numeric vector. Therefore the user has to
conform to the expected order. See 'Details' for further explanation.
\end{Description}
%
\begin{Usage}
\begin{verbatim}
opsr_check_start(start, W, Xs)
\end{verbatim}
\end{Usage}
%
\begin{Arguments}
\begin{ldescription}
\item[\code{start}] vector of starting values.

\item[\code{W}] matrix with explanatory variables for selection process.

\item[\code{Xs}] list of matrices with expalanatory varialbes for outcome process for each regime.
\end{ldescription}
\end{Arguments}
%
\begin{Details}
Expected order: 1. kappa threshold parameters (for ordered probit model),
2. parameters of the selection process (names starting with \code{s\_}), 3. parameters
of the outcome processes (names starting with \AsIs{\texttt{o[0-9]\_}}), 4. sigma, 5. rho.
If the same outcome process specification is used in the \code{formula}, the starting
values have to be repeated (i.e., the length of the \code{start} vector has to
correspond to the total number of estimated parameters in the model).
\end{Details}
%
\begin{Value}
Named numeric vector conforming to the expected structure.
\end{Value}
%
\begin{SeeAlso}
\code{\LinkA{opsr\_2step}{opsr.Rul.2step}}
\end{SeeAlso}

%%%%%

\inputencoding{utf8}
\HeaderA{opsr\_max\_threads}{Check Maximum Number of Threads Available}{opsr.Rul.max.Rul.threads}
%
\begin{Description}
Check Maximum Number of Threads Available
\end{Description}
%
\begin{Usage}
\begin{verbatim}
opsr_max_threads()
\end{verbatim}
\end{Usage}
%
\begin{Value}
integer
\end{Value}
%
\begin{SeeAlso}
\code{\LinkA{opsr\_check\_omp}{opsr.Rul.check.Rul.omp}}
\end{SeeAlso}

%%%%%

\inputencoding{utf8}
\HeaderA{opsr\_null\_model}{Null Model for OPSR Model fits}{opsr.Rul.null.Rul.model}
%
\begin{Description}
Intercept-only model with no error correlation.
\end{Description}
%
\begin{Usage}
\begin{verbatim}
opsr_null_model(object, ...)
\end{verbatim}
\end{Usage}
%
\begin{Arguments}
\begin{ldescription}
\item[\code{object}] an object of class \code{"opsr"}.

\item[\code{...}] further arguments passed to \code{\LinkA{opsr}{opsr}}.
\end{ldescription}
\end{Arguments}
%
\begin{Value}
An object of class \AsIs{\texttt{"opsr.null" "opsr"}}.
\end{Value}
%
\begin{Examples}
\begin{ExampleCode}
sim_dat <- opsr_simulate()
dat <- sim_dat$data
model <- ys | yo ~ xs1 + xs2 | xo1 + xo2
fit <- opsr(model, dat)
fit_null <- opsr_null_model(fit)
summary(fit_null)
\end{ExampleCode}
\end{Examples}

%%%%%

\inputencoding{utf8}
\HeaderA{opsr\_prepare\_coefs}{Prepares Coefficients for Likelihood Function}{opsr.Rul.prepare.Rul.coefs}
%
\begin{Description}
Extracts the coefficients for each regime
\end{Description}
%
\begin{Usage}
\begin{verbatim}
opsr_prepare_coefs(theta, nReg)
\end{verbatim}
\end{Usage}
%
\begin{Arguments}
\begin{ldescription}
\item[\code{theta}] named coefficient vector as parsed from formula interface \code{\LinkA{opsr}{opsr}}.

\item[\code{nReg}] integer number of regimes.
\end{ldescription}
\end{Arguments}
%
\begin{Value}
Named list of length \code{nReg}
\end{Value}
%
\begin{Examples}
\begin{ExampleCode}
sim_dat <- opsr_simulate()
dat <- sim_dat$data
model <- ys | yo ~ xs1 + xs2 | xo1 + xo2
start <- opsr(model, dat, .get2step = TRUE)
opsr_prepare_coefs(start, 3)
\end{ExampleCode}
\end{Examples}

%%%%%

\inputencoding{utf8}
\HeaderA{opsr\_simulate}{Simulate Data from an OPSR Process}{opsr.Rul.simulate}
%
\begin{Description}
Simulates data from an ordered probit process and separate (for each regime)
OLS process where the errors follow a multivariate normal distribution.
\end{Description}
%
\begin{Usage}
\begin{verbatim}
opsr_simulate(nobs = 1000, sigma = NULL)
\end{verbatim}
\end{Usage}
%
\begin{Arguments}
\begin{ldescription}
\item[\code{nobs}] number of observations to simulate.

\item[\code{sigma}] the covariance matrix of the multivariate normal.
\end{ldescription}
\end{Arguments}
%
\begin{Details}
Three ordinal outcomes are simulated and the distinct design matrices (\code{W} and
\code{X}) are used (if \code{W == X} the model is poorely identified). Variables \code{ys} and
\code{xs} in \code{data} correspond to the selection process and \code{yo}, \code{xo} to the outcome
process.
\end{Details}
%
\begin{Value}
Named list:
\begin{ldescription}
\item[\code{params}] ground truth parameters.
\item[\code{data}] simulated data (as observed by the researcher). See also 'Details' section.
\item[\code{errors}] error draws from the multivariate normal (as used in the latent
process).
\item[\code{sigma}] assumed covariance matrix (to generate \code{errors}).
\end{ldescription}
\end{Value}

%%%%%

\inputencoding{utf8}
\HeaderA{opsr.fit}{Fitter Function for Ordered Probit Switching Regression Models}{opsr.fit}
%
\begin{Description}
This is the basic computing engine called by \code{\LinkA{opsr}{opsr}} used to fit ordinal
probit switching regression models. Should usually \emph{not} be used directly.
The log-likelihood function is implemented in C++ which yields a considerable
speed-up. Parallel computation is implemented using \code{OpenMP}.
\end{Description}
%
\begin{Usage}
\begin{verbatim}
opsr.fit(
  Ws,
  Xs,
  Ys,
  start,
  fixed,
  weights,
  method,
  iterlim,
  printLevel,
  nThreads,
  .useR = FALSE,
  .loglik = FALSE,
  ...
)
\end{verbatim}
\end{Usage}
%
\begin{Arguments}
\begin{ldescription}
\item[\code{Ws}] list of matrices with explanatory variables for selection process for each regime.

\item[\code{Xs}] list of matrices with expalanatory varialbes for outcome process for each regime.

\item[\code{Ys}] list of vectors with continuous outcomes for each regime.

\item[\code{start}] a numeric vector with the starting values (passed to \code{maxLik::maxLik}).

\item[\code{fixed}] parameters to be treated as constants at their \code{start} values. If
present, it is treated as an index vector of \code{start} parameters (passed to \code{maxLik::maxLik}).

\item[\code{weights}] a vector of weights to be used in the fitting process. Has to
conform with order (\code{w <- weights[order(Z)]}, where Z is the ordinal
outcome).

\item[\code{method}] maximzation method (passed to \code{maxLik::maxLik}).

\item[\code{iterlim}] maximum number of iterations (passed to \code{maxLik::maxLik}).

\item[\code{printLevel}] larger number prints more working information (passed to \code{maxLik::maxLik}).

\item[\code{nThreads}] number of threads to be used. Do not pass higher number than
number of ordinal outcomes. See also \code{\LinkA{opsr\_check\_omp}{opsr.Rul.check.Rul.omp}} and \code{\LinkA{opsr\_max\_threads}{opsr.Rul.max.Rul.threads}}.

\item[\code{.useR}] if \code{TRUE}, usese \code{\LinkA{loglik\_R}{loglik.Rul.R}}. Go grab a coffe.

\item[\code{.loglik}] if \code{TRUE}, returns the vector of log-likelihood values given
the parameters passed via \code{start}.

\item[\code{...}] further arguments passed to \code{maxLik::maxLik}.
\end{ldescription}
\end{Arguments}
%
\begin{Value}
object of class \AsIs{\texttt{"maxLik" "maxim"}}.
\end{Value}
%
\begin{SeeAlso}
\code{maxLik::maxLik}, \code{\LinkA{loglik\_cpp}{loglik.Rul.cpp}}, \code{\LinkA{opsr}{opsr}}
\end{SeeAlso}

%%%%%

\inputencoding{utf8}
\HeaderA{opsr}{Fitting Ordered Probit Switching Regression Models}{opsr}
%
\begin{Description}
High-level formula interface to the workhorse \code{\LinkA{opsr.fit}{opsr.fit}}.
\end{Description}
%
\begin{Usage}
\begin{verbatim}
opsr(
  formula,
  data,
  subset,
  weights,
  na.action,
  start = NULL,
  fixed = NULL,
  method = "BFGS",
  iterlim = 1000,
  printLevel = 2,
  nThreads = 1,
  .get2step = FALSE,
  .useR = FALSE,
  .censorRho = TRUE,
  .loglik = FALSE,
  ...
)
\end{verbatim}
\end{Usage}
%
\begin{Arguments}
\begin{ldescription}
\item[\code{formula}] an object of class \AsIs{\texttt{"Formula" "formula"}}: A symbolic description
of the model to be fitted. The details of model specification are given under
'Details'.

\item[\code{data}] an optional data frame, list or environment (or object coercible by
\code{as.data.frame} to a data frame) containing the variables in the model. If
not found in \code{data}, the variables are taken from \code{environment(formula)},
typically the environment from which \code{opsr} is called.

\item[\code{subset}] an optional vector specifying a subset of observations to be used
in the fitting process. (See additional details in the 'Details' section of
the \code{model.frame} documentation.).

\item[\code{weights}] an optional vector of weights to be used in the fitting process.
Should be \code{NULL} or a numeric vector. If non-NULL, then observation-specific
log-likelihood contributions are multiplied by their corresponding weight
before summing.

\item[\code{na.action}] a function which indicates what should happen when the data
contain \code{NA}s. The default is set by the \code{na.action} setting of \code{options},
and is \code{na.fail} if that is unset. The 'factory-fresh' default is \code{na.omit}.
Another possible value is \code{NULL}, no action. Value \code{na.exclude} can be useful.

\item[\code{start}] a numeric vector with the starting values (passed to \code{maxLik::maxLik}).
If no starting values are provided, reasonable values are auto-generated via
the Heckman 2-step procedure \code{\LinkA{opsr\_2step}{opsr.Rul.2step}}. The structure of \code{start} has to
conform with \code{opsr}'s expectations. See \code{\LinkA{opsr\_check\_start}{opsr.Rul.check.Rul.start}} for further details.

\item[\code{fixed}] parameters to be treated as constants at their \code{start} values. If
present, it is treated as an index vector of \code{start} parameters (passed to
\code{maxLik::maxLik}).

\item[\code{method}] maximzation method (passed to \code{maxLik::maxLik}).

\item[\code{iterlim}] maximum number of iterations (passed to \code{maxLik::maxLik}).

\item[\code{printLevel}] larger number prints more working information (passed to \code{maxLik::maxLik}).

\item[\code{nThreads}] number of threads to be used. Do not pass higher number than
number of ordinal outcomes. See also \code{\LinkA{opsr\_check\_omp}{opsr.Rul.check.Rul.omp}} and \code{\LinkA{opsr\_max\_threads}{opsr.Rul.max.Rul.threads}}.

\item[\code{.get2step}] if \code{TRUE}, returns starting values as generated by \code{\LinkA{opsr\_2step}{opsr.Rul.2step}}. Will
not proceed with the maximum likelihood estimation.

\item[\code{.useR}] if \code{TRUE} usese \code{\LinkA{loglik\_R}{loglik.Rul.R}}. Go grab a coffe.

\item[\code{.censorRho}] if \code{TRUE}, rho starting values are censored to lie in the
interval [-0.85, 0.85].

\item[\code{.loglik}] if \code{TRUE}, returns the vector of log-likelihood values given
the parameters passed via \code{start}.

\item[\code{...}] further arguments passed to \code{maxLik::maxLik}.
\end{ldescription}
\end{Arguments}
%
\begin{Details}
Models for \code{opsr} are specified symbolically. A typical model has the form
\code{ys | yo \textasciitilde{} terms\_s | terms\_o1 | terms\_o2 | ...}. \code{ys} is the ordered (numeric)
response vector (starting from 1, in integer-increasing fashion). For the \code{terms}
specification the rules of the regular formula interface apply (see also \code{stats::lm}).
The intercept in the \code{terms\_s} (selection process) is excluded automatically
(no need to specify \code{-1}). If the user wants to specify the same process for
all continuous outcomes, two processes are enough (\code{ys | yo \textasciitilde{} terms\_s | terms\_o}).
Note that the model is poorly identifiable if \code{terms\_s == terms\_o} (same regressors
are used in selection and outcome processes).
\end{Details}
%
\begin{Value}
An object of class \AsIs{\texttt{"opsr" "maxLik" "maxim"}}.
\end{Value}
%
\begin{Examples}
\begin{ExampleCode}
## simulated data
sim_dat <- opsr_simulate()
dat <- sim_dat$data  # 1000 observations
sim_dat$sigma  # cov matrix of errors
sim_dat$params  # ground truth

## specify a model
model <- ys | yo ~ xs1 + xs2 | xo1 + xo2 | xo1 + xo2 | xo1 + xo2
## since we use the same specification...
model <- ys | yo ~ xs1 + xs2 | xo1 + xo2

## estimate
fit <- opsr(model, dat)

## inference
summary(fit)

## using update and model comparison
## only intercepts for the continuous outcomes
fit_updated <- update(fit, ~ . | 1)
## null model
fit_null <- opsr_null_model(fit)

## likelihood ratio test
anova(fit_null, fit_updated, fit)

## predict
p1 <- predict(fit, group = 1, type = "response")
p2 <- predict(fit, group = 1, counterfact = 2, type = "response")
plot(p1, p2)
abline(a = 0, b = 1, col = "red")

## produce formatted tables
texreg::screenreg(fit, beside = TRUE, include.pseudoR2 = TRUE,
                  include.R2 = TRUE)

\end{ExampleCode}
\end{Examples}

%%%%%

\inputencoding{utf8}
\HeaderA{predict.opsr}{Predict Method for OPSR Model Fits}{predict.opsr}
%
\begin{Description}
Obtains predictions for the selection process (probabilities), the outcome process,
or returns the inverse mills ratio. Handles also log-transformed outcomes.
\end{Description}
%
\begin{Usage}
\begin{verbatim}
## S3 method for class 'opsr'
predict(
  object,
  newdata,
  group,
  counterfact = NULL,
  type = c("response", "unlog-response", "prob", "mills",
           "correction", "Xb"),
  delta = 1,
  ...
)
\end{verbatim}
\end{Usage}
%
\begin{Arguments}
\begin{ldescription}
\item[\code{object}] an object of class \code{"opsr"}.

\item[\code{newdata}] an optional data frame in which to look for variables used in
\code{object\$formula}. See also \code{\LinkA{model.matrix.opsr}{model.matrix.opsr}}.

\item[\code{group}] predict outcome of this group (regime).

\item[\code{counterfact}] counterfactual group.

\item[\code{type}] type of prediction. Can be abbreviated. See 'Details' section for
more information.

\item[\code{delta}] constant that was added during the continuity correction
(\eqn{log(Y_j + \delta)}{}). Only applies for \code{type = "unlog-response"}.

\item[\code{...}] further arguments passed to or from other methods.
\end{ldescription}
\end{Arguments}
%
\begin{Details}
Elements are \code{NA\_real\_} if the \code{group} does not correspond to the observed
regime (selection outcome). This ensures consistent output length.

If the \code{type} argument is \code{"response"} then the continuous outcome is predicted.
Use \code{"unlog-response"} if the outcome response was log-transformed (i.e., either
in the \code{formula} specification or during data pre-processing).
\code{"prob"} returns the probability vector of belonging to \code{group}, \code{"mills"}
returns the inverse mills ratio, \code{"correction"} the heckman correction (i.e.,
\eqn{\rho_j * \sigma_j * \text{mills}}{}) and \code{"Xb"} returns \eqn{X \beta}{}.
\end{Details}
%
\begin{Value}
a vector of length \code{nrow(newdata)} (or data used during estimation).
\end{Value}
%
\begin{SeeAlso}
\code{stats::predict}
\end{SeeAlso}
%
\begin{Examples}
\begin{ExampleCode}
sim_dat <- opsr_simulate()
dat <- sim_dat$data
model <- ys | yo ~ xs1 + xs2 | xo1 + xo2
fit <- opsr(model, dat)
p <- predict(fit, group = 1, type = "response")

fit_log <- update(fit, . | log(yo) ~ .)
p_unlog <- predict(fit, group = 1, type = "unlog-response")
\end{ExampleCode}
\end{Examples}

%%%%%

\inputencoding{utf8}
\HeaderA{print.anova.opsr}{Print Method for ANOVA OPSR Objects}{print.anova.opsr}
%
\begin{Description}
Print Method for ANOVA OPSR Objects
\end{Description}
%
\begin{Usage}
\begin{verbatim}
## S3 method for class 'anova.opsr'
print(
  x,
  digits = max(getOption("digits") - 2L, 3L),
  signif.stars = getOption("show.signif.stars"),
  print.formula = TRUE,
  ...
)
\end{verbatim}
\end{Usage}
%
\begin{Arguments}
\begin{ldescription}
\item[\code{x}] an object of class \code{"anova.opsr"}.

\item[\code{digits}] minimal number of \emph{significant} digits, see \code{print.default}.

\item[\code{signif.stars}] if \code{TRUE}, P-values are additionally encoded visually
as 'significance stars' in order to help scanning of long coefficient tables.
It defaults to the \code{show.signif.stars} slot of \code{options}.

\item[\code{print.formula}] if \code{TRUE}, the formulas of the models are printed.

\item[\code{...}] further arguments passed to \code{stats::printCoefmat}.
\end{ldescription}
\end{Arguments}
%
\begin{Value}
Prints tables in a 'pretty' form and returns \code{x} invisibly.
\end{Value}
%
\begin{SeeAlso}
\code{stats::printCoefmat}, \code{\LinkA{anova.opsr}{anova.opsr}}
\end{SeeAlso}

%%%%%

\inputencoding{utf8}
\HeaderA{print.summary.opsr}{Print Method for Summary OPSR Objects}{print.summary.opsr}
%
\begin{Description}
Print Method for Summary OPSR Objects
\end{Description}
%
\begin{Usage}
\begin{verbatim}
## S3 method for class 'summary.opsr'
print(x, digits = max(3L, getOption("digits") - 3L),
      print.call = TRUE, ...)
\end{verbatim}
\end{Usage}
%
\begin{Arguments}
\begin{ldescription}
\item[\code{x}] and object of class \code{"summary.opsr"}

\item[\code{digits}] minimum number of significant digits to be used for most numbers (passed to \code{stats::printCoefmat}).

\item[\code{print.call}] if \code{TRUE}, prints the underlying \code{\LinkA{opsr}{opsr}} call.

\item[\code{...}] further arguments passed to or from other methods.
\end{ldescription}
\end{Arguments}
%
\begin{Value}
Prints summary in 'pretty' form and returns \code{x} invisibly.
\end{Value}
%
\begin{SeeAlso}
\code{stats::printCoefmat}, \code{\LinkA{summary.opsr}{summary.opsr}}
\end{SeeAlso}

%%%%%

\inputencoding{utf8}
\HeaderA{summary.opsr}{Summarizing OPSR Model Fits}{summary.opsr}
%
\begin{Description}
Follows the convention that \code{\LinkA{opsr}{opsr}} does the bare minimum model fitting and
inference is performed in \code{summary}.
\end{Description}
%
\begin{Usage}
\begin{verbatim}
## S3 method for class 'opsr'
summary(object, rob = TRUE, ...)
\end{verbatim}
\end{Usage}
%
\begin{Arguments}
\begin{ldescription}
\item[\code{object}] an object of class \code{"opsr"}.

\item[\code{rob}] if \code{TRUE}, the \code{sandwich::sandwich} covariance matrix extimator is used.

\item[\code{...}] further arguments passed to or from other methods.
\end{ldescription}
\end{Arguments}
%
\begin{Value}
An object of class \code{"summary.opsr"}.
In particular the elements \code{GOF}, \code{GOFcomponents} and \code{wald} require further
explanation:
\begin{ldescription}
\item[\code{GOF}] Contains the conventional \emph{goodness of fit} indicators for the full
model. \code{LL2step} is the log-likelihood of the Heckman two-step solution (if
the default starting values were used). \code{LLfinal} is the log-likelihood at
final convergence and \code{AIC}, \code{BIC} the corresponding information critereon.
\item[\code{GOFcomponents}] Contains the \emph{goodness of fit}  for the model components.
\code{LLprobit} is the log-likelihood (LL) contribution of the ordered probit model.
\code{LLprobitEl} the LL of the "equally likely" and \code{LLprobitMs} the LL of the
"market share" model. With these three metrics the pseudo R2 is computed and
returned as \code{pseudoR2el} and \code{pseudoR2ms}. \code{R2} reports the usual coefficient
of determination (for the continuous outcomes jointly and for each regime
separately).
\item[\code{wald}] Contains the results of two \emph{Wald-tests} as conducted with help
of \code{car::linearHypothesis}. The two H0 hypothesis are 1. All coefficients
of the explanatory variables are 0 and 2. The rho parameters (capturing error
correlation) are zero.
\end{ldescription}
\end{Value}

%%%%%

\inputencoding{utf8}
\HeaderA{telework\_data}{Telework Data}{telework.Rul.data}
\keyword{datasets}{telework\_data}
%
\begin{Description}
Telework data as used in \citet{Wang+Mokhtarian:2024}.
\end{Description}
%
\begin{Usage}
\begin{verbatim}
telework_data
\end{verbatim}
\end{Usage}
%
\begin{Format}
Data frame with numeric columns
\begin{description}

\item[id] Respondent ID
\item[weight] Sample weight
\item[vmd] Weekly vehicle-miles traveled
\item[vmd\_ln] Log-transformed VMD, the dependent variable of the outcome model
\item[twing\_status] Teleworking status: 1=Non-TWer, 2=Non-usual TWer, 3=Usual TWer
\item[female] Sex: female
\item[age\_mean] Mean-centered age
\item[age\_mean\_sq] Sqaure of mean-centered age
\item[race\_white] Race: white only
\item[race\_black] Race: black only
\item[race\_other] Race: other
\item[edu\_1] Education: high school or lower
\item[edu\_2] Education: some college
\item[edu\_3] Education: BA or higher
\item[hhincome\_1] Household income: less than \$50,000
\item[hhincome\_2] Household income: \$50,000 to \$99,999
\item[hhincome\_3] Household income: \$100,000 or more
\item[flex\_work] Flexible work schedule
\item[work\_fulltime] Full-time worker
\item[twing\_feasibility] Teleworking feasibility (days/month)
\item[vehicle] Number of household vehicles
\item[child] Number of children
\item[urban] Residential location: urban
\item[suburban] Residential location: suburban
\item[smalltown] Residential location: small town
\item[rural] Residential location: rural
\item[att\_prolargehouse] Attitude: pro-large-house
\item[att\_proactivemode] Attitude: pro-active-mode
\item[att\_procarowning] Attitude: pro-car-owning
\item[att\_wif] Attitude: work-interferes-with-family
\item[att\_proteamwork] Attitude: pro-teamwork
\item[att\_tw\_effective\_teamwork] Attitude: TW effective teamwork
\item[att\_tw\_enthusiasm] Attitude: TW enthusiasm
\item[att\_tw\_location\_flex] Attitude: TW location flexibility
\item[region\_waa] Region indicator: respondents from WAA MSA

\end{description}

\end{Format}
%
\begin{References}
\bibentry{Wang+Mokhtarian:2024}
\end{References}
%
\begin{Examples}
\begin{ExampleCode}


## model as in Xinyi & Mokhtarian (2024)
f <-
  ## ordinal and continuous outcome
  twing_status | vmd_ln ~
  ## selection model
  edu_2 + edu_3 + hhincome_2 + hhincome_3 +
  flex_work + work_fulltime + twing_feasibility +
  att_proactivemode + att_procarowning +
  att_wif + att_proteamwork +
  att_tw_effective_teamwork + att_tw_enthusiasm + att_tw_location_flex |
  ## outcome model NTW
  female + age_mean + age_mean_sq +
  race_black + race_other +
  vehicle + suburban + smalltown + rural +
  work_fulltime +
  att_prolargehouse + att_procarowning +
  region_waa |
  ## outcome model NUTW
  edu_2 + edu_3 + suburban + smalltown + rural +
  work_fulltime +
  att_prolargehouse + att_proactivemode + att_procarowning |
  ## outcome model UTW
  female + hhincome_2 + hhincome_3 +
  child + suburban + smalltown + rural +
  att_procarowning +
  region_waa

fit <- opsr(f, telework_data)
texreg::screenreg(fit, beside = TRUE, include.pseudoR2 = TRUE,
                  include.R2 = TRUE)


\end{ExampleCode}
\end{Examples}

%%%%%

\inputencoding{utf8}
\HeaderA{timeuse\_data}{TimeUse+ Data}{timeuse.Rul.data}
\keyword{datasets}{timeuse\_data}
%
\begin{Description}
TimeUse+ \citep{Winkler+Meister+Axhausen:2024} was a tracking study
conducted at the Institute for Transport Planning and Systems (IVT) at ETH Zurich.
The data allows researchers to investigate time-use and mobility patterns.
Full data access can be requested via \url{doi:10.3929/ethz-b-000634868}.
\end{Description}
%
\begin{Usage}
\begin{verbatim}
timeuse_data
\end{verbatim}
\end{Usage}
%
\begin{Format}
Data frame with numeric and factor columns
\begin{description}

\item[id] Respondent ID
\item[start\_tracking] Indicator for the months when the participant started tracking
\item[weekly\_km] Weekly distance covered (in km) across all modes
\item[log\_weekly\_km] Log of \code{weekly\_km}
\item[wfh\_days] Based on tracked work episodes. Average full working days spent in home office during a typical week.
\item[wfh] Derived from \code{wfh\_days}: NTW=Non-TWer, NUTW=Non-usual TWer (< 3 days/week), UTW=Usual TWer(3+ days/week)
\item[commute\_km] Map matched commute distance in km
\item[log\_commute\_km] Log of \code{commute\_km}
\item[age] Age
\item[car\_access] Has access to a household car
\item[dogs] Household owns dogs
\item[driverlicense] Owns driver's license for cars
\item[educ\_higher] Higher education (e.g., university)
\item[fixed\_workplace] Main work location does not change regularly
\item[grocery\_shopper] Does usually do the grocery shopping
\item[hh\_income] Gross household income per month (CHF)
\item[hh\_size] Total household size
\item[isco\_clerical] International standard classification of Occupations (ISCO-08): Clerical support workers
\item[isco\_craft] International standard classification of Occupations (ISCO-08): Craft related trade workers
\item[isco\_elementary] International standard classification of Occupations (ISCO-08): Elementary occupations
\item[isco\_managers] International standard classification of Occupations (ISCO-08): Managers
\item[isco\_plant] International standard classification of Occupations (ISCO-08): Plant and machine operators, and assemblers
\item[isco\_professionals] International standard classification of Occupations (ISCO-08): Professionals
\item[isco\_service] International standard classification of Occupations (ISCO-08): Service and sales workers
\item[isco\_agri] International standard classification of Occupations (ISCO-08): Skilled agricultural, forestry and fishery workers
\item[isco\_tech] International standard classification of Occupations (ISCO-08): Technicians and associate professionals
\item[married] Married
\item[n\_children] Number of household members below the age of 18
\item[freq\_onl\_order] Orders products online more than once a month
\item[parking\_home] Has at least one reserved parking space at home
\item[parking\_work] Has at least one reserved parking space at work location
\item[permanent\_employment] Employment contract type: permenent (unlimited employed)
\item[rents\_home] Residential situation: Rents home
\item[res\_loc] Residential location
\item[sex\_male] Gender
\item[shift\_work] Whether participant works in shifts
\item[swiss] Swiss citizen
\item[vacation] Participant took time off of work during study
\item[workload] Workload (\% of full-time employment which is 41.7 h/week)
\item[young\_kids] Whether children aged 12 or younger live in the household

\end{description}

\end{Format}
%
\begin{Details}
The data comprises employed individuals only and is based on valid days only.
A valid day has at least 20h of information where 70\% of the events were
validated by the user. The telework variables are based on tracked work activities.
\end{Details}
%
\begin{References}
\bibentry{Winkler+Meister+Axhausen:2024}
\end{References}

