\RequirePackage{fix-cm}
\documentclass[%
    twoside, openright, titlepage, numbers=noenddot,%
    cleardoublepage=empty,%
    abstract=false,%
    BCOR=5.5mm, paper=a5, fontsize=10pt,% A5 soft cover
    %BCOR=5.5mm, paper=17cm:24cm, fontsize=10pt,% 17 cm x 24 cm
    %BCOR=5mm, paper=15.59cm:23.39cm, fontsize=10pt,% Royal soft cover
    %BCOR=0mm, paper=15.24cm:22.86cm, fontsize=10pt,% US-Trade hard cover
]{mythesis}

\input{preamble/general}
\usepackage[final]{pdfpages}

% Custom commands
%% -- org ----------------------------------------------------------------------
\newcommand{\todo}[1]{%
\reversemarginpar
\marginpar{TODO}
#1
\marginpar{----}
\normalmarginpar
}

%% for underlying paper, contributions and changes
\newenvironment{ChapterInfoTable}
{
\begin{longtable}{p{3.5cm}p{7.3cm}}
}
{
\end{longtable}
}

\newcommand{\ChapterInfoEntry}[2]{
\small{\emph{#1}} & #2 \\
\addlinespace
}

%% -- opsr ---------------------------------------------------------------------
\newcommand{\jth}{\ensuremath{j^{\mathrm{th}}\,}}  % jth will never follow by punctuation
\newcommand{\Xb}{\boldsymbol{X_j}\boldsymbol{\beta_j}}
\newcommand{\Xbd}{\boldsymbol{X_{j'}}\boldsymbol{\beta_{j'}}}
\newcommand{\Wg}{\boldsymbol{W}\boldsymbol{\gamma}}


% Custom packages
%% this line below is required (commented out) if you define your own Sinput and
%% Soutput => see Sweave user manual
% \usepackage{Sweave}

%% recommended packages
\usepackage{orcidlink,thumbpdf}

%% another package (only for the draft article)
\usepackage{framed}

\usepackage{amsmath}
\usepackage{booktabs}
\usepackage{dcolumn}
\usepackage{array}
\usepackage{graphicx}
\usepackage{pdflscape}
\usepackage{longtable}

\usepackage{multirow}
\usepackage{threeparttable}
\usepackage[table,xcdraw]{xcolor}


% Bibliography
% \addbibresource[label=ownpubs]{ownpubs.bib}
% \addbibresource{bibliography.bib}
% \addbibresource{misc.bib}
\bibliography{ownpubs.bib}
\bibliography{bibliography.bib}
\bibliography{misc.bib}




\usepackage{Sweave}
\begin{document}
\frenchspacing
\raggedbottom%
\selectlanguage{english}
\pagenumbering{roman}
\pagestyle{scrplain}

%
% Cover
%
% Uncomment and adapt these lines if you want to include a cover PDF.
%
\includepdf[pages={1,{}}]{cover/crop/cover_front.pdf}
\cleardoublepage\setcounter{page}{1}

%
% Frontmatter
%

%*******************************************************
% Little Dirty Titlepage
%*******************************************************
\thispagestyle{empty}
%*******************************************************
\begin{center}
    \spacedlowsmallcaps{\myName} \\ \medskip                        

    \begingroup
        \color{chapter-color}\spacedallcaps{\myTitle}
    \endgroup
\end{center}        
%*******************************************************
% Titlepage
%*******************************************************
\begin{titlepage}
	% if you want the titlepage to be centered, uncomment and fine-tune the line below (KOMA classes environment)
	%\begin{addmargin}[-1cm]{-3cm}
    \begin{center}
        \large
        \begingroup
            \spacedlowsmallcaps{Diss. ETH No. \myDissNumber}
        \endgroup

        \hfill

        \vfill

        \begingroup
            \spacedallcaps{\myTitle}
            %\spacedallcaps{\myTitleLineOne}\\
            %\spacedallcaps{\myTitleLineTwo}\\
            %\spacedallcaps{\myTitleLineThree}
        \endgroup

        \vfill

        \begingroup
            A dissertation submitted to attain the degree of\\
            \vspace{0.5em}
            \spacedlowsmallcaps{Doctor of Sciences}
            of
            \spacedlowsmallcaps{ETH Zurich} \\
            (Dr.\ sc.\ ETH Zurich)
        \endgroup

        \vfill

        \begingroup
            presented by\\
            \vspace{0.5em}
            \spacedlowsmallcaps{\myName}\\
            MSc. Economics \\ Stockholm School of Economics\\
            \vspace{0.5em}
            born on 22 August 1992\\
            citizen of Switzerland
        \endgroup

        \vfill

        \begingroup
            accepted on the recommendation of\\
            \vspace{0.5em}
            Prof.\ Dr.\ D. Kaufmann, examiner\\
            Prof.\ em.\ Dr.\ K. W. Axhausen co-examiner\\
            Prof.\ Dr.\ T. Bernauer, co-examiner\\
            Prof.\ Dr.\ M. Bierlaire, co-examiner\\
            Prof.\ Dr.\ S. Hess, co-examiner
        \endgroup

        \vfill

        \myTime%

        \vfill
    \end{center}
  %\end{addmargin}
\end{titlepage}
\thispagestyle{empty}

\hfill

\vfill

\noindent\myName: \textit{\myTitle,} %\mySubtitle, %\myDegree, 
\textcopyright\ \myTime

\bigskip

\noindent\spacedlowsmallcaps{DOI}: \myDOI

%\bigskip
%
%\noindent\spacedlowsmallcaps{Supervisors}: \\
%\myProf \\
%\myOtherProf \\ 
%\mySupervisor
%
%\medskip
%
%\noindent\spacedlowsmallcaps{Location}: \\
%\myLocation
%
%\medskip
%
%\noindent\spacedlowsmallcaps{Time Frame}: \\
%\myTime
\cleardoublepage%
%*******************************************************
% Dedication
%*******************************************************
\thispagestyle{empty}
%\phantomsection
\refstepcounter{dummy}
%\pdfbookmark[1]{Dedication}{Dedication}

\vspace*{3cm}

\begin{center}
    To Mileva
\end{center}

\medskip
\cleardoublepage%
%*******************************************************
% Abstract
%*******************************************************
%\renewcommand{\abstractname}{Abstract}
\pdfbookmark[1]{Abstract}{Abstract}
\begingroup
\let\clearpage\relax
\let\cleardoublepage\relax
\let\cleardoublepage\relax

\chapter*{Abstract}

English abstract here.

\endgroup

\cleardoublepage%

\begingroup
\let\clearpage\relax
\let\cleardoublepage\relax
\let\cleardoublepage\relax

\begin{otherlanguage}{ngerman}
\pdfbookmark[1]{Zusammenfassung}{Zusammenfassung}
\chapter*{Zusammenfassung}

Deutsche Zusammenfassung hier.

\end{otherlanguage}

\endgroup

\vfill
\cleardoublepage%
%*******************************************************
% Acknowledgments
%*******************************************************
\pdfbookmark[1]{Acknowledgements}{acknowledgements}

\bigskip

\begingroup
\let\clearpage\relax
\let\cleardoublepage\relax
\let\cleardoublepage\relax
\chapter*{Acknowledgements}

\def\thanks#1{%
\begingroup
\leftskip1em
\noindent #1
\par
\endgroup
}

I would like to thank \dots

\endgroup
\pagestyle{scrheadings}
\cleardoublepage%
%*******************************************************
% Table of Contents
%*******************************************************
%\phantomsection
\refstepcounter{dummy}
\pdfbookmark[1]{\contentsname}{tableofcontents}
\setcounter{tocdepth}{2} % <-- 2 includes up to subsections in the ToC
\setcounter{secnumdepth}{3} % <-- 3 numbers up to subsubsections
\manualmark%
\markboth{\spacedlowsmallcaps{\contentsname}}{\spacedlowsmallcaps{\contentsname}}
\tableofcontents
\automark[section]{chapter}
\renewcommand{\chaptermark}[1]{\markboth{\spacedlowsmallcaps{#1}}{\spacedlowsmallcaps{#1}}}
\renewcommand{\sectionmark}[1]{\markright{\thesection\enspace\spacedlowsmallcaps{#1}}}
%*******************************************************
% List of Figures and of the Tables
%*******************************************************
\clearpage

\begingroup
    \let\clearpage\relax
    \let\cleardoublepage\relax
    \let\cleardoublepage\relax
    %*******************************************************
    % List of Figures
    %*******************************************************
    %\phantomsection
    % \refstepcounter{dummy}
    %\addcontentsline{toc}{chapter}{\listfigurename}
    % \pdfbookmark[1]{\listfigurename}{lof}
    % \listoffigures

    % \vspace{8ex}

    %*******************************************************
    % List of Tables
    %*******************************************************
    %\phantomsection
    % \refstepcounter{dummy}
    %\addcontentsline{toc}{chapter}{\listtablename}
    % \pdfbookmark[1]{\listtablename}{lot}
    % \listoftables

    % \vspace{8ex}
    % \newpage

    %*******************************************************
    % List of Listings
    %*******************************************************
      %\phantomsection
    %\refstepcounter{dummy}
    %\addcontentsline{toc}{chapter}{\lstlistlistingname}
    %\pdfbookmark[1]{\lstlistlistingname}{lol}
    %\lstlistoflistings%

    %\vspace{8ex}

    % Notation
    \refstepcounter{dummy}
    \pdfbookmark[1]{Notation}{notation}
    \markboth{\spacedlowsmallcaps{Notation}}{\spacedlowsmallcaps{Notation}}
    \chapter*{Notation}

    \section*{Frequently used abbreviations}%
    \vskip -2em
    \begin{tabularx}{\textwidth}{lX}
      %\toprule%
      %\tableheadline{Symbol} & \tableheadline{Meaning} \\
      %\midrule%
      TW & telework \\
      TWers & teleworkers \\
      TWing & teleworking \\
      NTW & non-telework (0 days/week) \\
      NUTW & non-usual telework (<3 days/week) \\
      UTW & usual telework (3+ days/week) \\
      OPSR & ordinal probit switching regression \\
      TE & treatment effect \\
      ATE & average treatment effect \\
      %\bottomrule
    \end{tabularx}

    \section*{Physical constants}
    \sisetup{separate-uncertainty=false}
    \vskip -2em
    \begin{tabularx}{\textwidth}{lX}
$c$ & speed of light in vacuum, $c=\u{299792458}{\metre\per\second}$ \\
    \end{tabularx}
    \begin{flushright}
    (CODATA 2014~\cite{codata})
    \end{flushright}
    \sisetup{separate-uncertainty=true}
\endgroup

%
% Mainmatter
%
\cleardoublepage\pagenumbering{arabic}%
\def\dir{chapters/introduction}
\usepackage{natbib}
% citation style
\providecommand{\mybibstyle}{
  \setcitestyle{authoryear,round}
  \bibliographystyle{preamble/natbib/ivt-eng.bst}
  }
\mybibstyle


\cleardoublepage%
\chapter{Modal splits before during and after the pandemic in Switzerland}
\label{ch:mpp}

\dictum[John Doe]{%
  Hello world!}%
\vskip 1em


The adjustments of mobility patterns during early stages of the pandemic are well understood. However, various effects are intertwined in these observations and therefore the findings' robustness remains questionable. The MOBIS-Covid data set provides a unique opportunity to put these initial findings in perspective as a large panel has been tracked from before the crisis up until today. Switzerland lifted all its measures counteracting the spread in mid February 2022, reaching a potential new equilibrium in the months that followed. A comprehensive set of descriptive indicators has been constructed for Switzerland in order to disentangle the narrative of the crisis from the perspective of transport demand. Special emphasis has been given to the question of how new hybrid working arrangements influence modal splits (as measured by mode distance shares). The descriptive findings are strengthened by a mixed multiple discrete-continuous extreme value model (MMDCEV). We find a shift in modal splits away from car and train, where in particular bus could expand its mode share most pronounced. While the cycling and walking booms were temporary, the bicycle still shows sightly higher mode distance shares (driven by more rather than longer trips). While there are observable differences between the different working arrangements (home office, mixture and in office), we do not find significant model parameters when allowing for unobserved heterogeneity. This suggests that the working arrangement segments the population along socioeconomic dimensions with different mobility behaviour and mode preferences. Modes are more satiated in the post-pandemic world indicating that people use fewer modes in their weekly modal mix but use them more intensely. However, the model suggests that the pandemic should not be read as a structural break in mode preferences. We will therefore likely see a further normalization in the months that come.

This chapter consists of the following peer-reviewed journal paper:

The contributions for this paper are as follows:

\begin{leftbar}
\begin{itemize}
\item See also thesis by Reck...
\item Maybe add \emph{The following adjusmtents were made to the published text...} or similar.
\item Namespace labels.
\end{itemize}
\end{leftbar}

\section{Introduction}
\label{sec:introduction}

The COVID Pandemic hit our system from various angles: First and foremost, it was a health crisis with risk asymmetries between different groups in society. It was a considerable shock to our economies with high uncertainty and supply chains breaking down. Firms have proven to be very flexible, adopting working from home, keeping productivity at high levels. Last but not least, the initial solidarity seemed to fade with the duration of the crisis and political polarization peaked with the vaccine roll-out. Not surprisingly, almost all of these dimensions played into our mobility behavior in one way or the other, either directly (e.g. because of health concerns) or indirectly (e.g. by reducing commuting activities through working from home).

Several studies have been trying to describe the evolution of transport demand during early stages of the crisis and by different means such as tracking studies or questionnaires. But of course, all the effects outlined above, were intertwined and a clear attribution was difficult. Therefore, the question of whether or not these patterns persist in the post-pandemic world could not be answered.

The MOBIS-Covid study \citep{Molloy+Etal:2022} provides a unique opportunity to look at the persistence of these effects. A Swiss panel with comparably rich socioeconomic information has been tracked from before the crisis until today. In Switzerland, all restrictions and measures to contain the spread were lifted on February 17, 2022, and we potentially converged to a new equilibrium in the months that followed. Therefore, we have revisited the MOBIS-Covid data and tried to comment on how sustainable initial findings have proven to be and translate to recent times. For this purpose, a comprehensive set of descriptive indicators has been constructed in order to disentangle the narrative of the crisis. Special emphasis is given to the question of how new hybrid working arrangements (i.e. home office) drive modal splits (as measured by mode distance shares). The descriptive findings are strengthened by a mixed multiple discrete-continuous extreme value model (MMDCEV) where we model both the discrete and continuous dimension of weekly mode distance shares and compare the pre-pandemic to the arguably post-pandemic world.

The remainder of the text is structured as follows: In Section~\ref{sec:literature} we review the studies that explore the impact of COVID on mobility patterns with a focus on modal shifts. We then introduce the MOBIS-Covid sample and data collection process in Section~\ref{sec:data}. Section~\ref{sec:descriptive-analysis} is the core of the paper and discusses the narrative of the pandemic with various descriptive indicators. The MMDCEV modeling framework is introduced in Section~\ref{sec:mmdcev}. Results are presented and discussed in Section~\ref{sec:results}, while Section~\ref{sec:conclusion} concludes and provides an outlook.

\subsection{Covid-19 timeline in Switzerland}
\label{sec:timeline}

Based both on the COVID case numbers and the measures imposed to counteract the spread of the virus, the following six phases can be distinguished. The phases are visualized together with COVID case numbers (rolling 14-day mean) in Figure~\ref{fig:participation-and-casenumbers}. The narrative of the pandemic is presented hereafter and summarized in Table~\ref{tab:phases}, where each phase of expansion (constraining the free movement) is followed by a phase of relaxation.

\emph{Phase 1 (restriction):} COVID-19 reached Switzerland in early 2020. The situation deteriorated quickly and by March 20, 2020 over $4800$ people infected. This first phase can thus be characterized by a rapid spread of the virus as well as high uncertainty. On March 16, 2020 the Federal Council declared an extraordinary situation, which allowed it to introduce uniform measures in all cantons. A lockdown was imposed, closing all non-essential businesses, along with schools, recreational facilities and public parks. As a consequence, employers were urged to reorganize the working hours of their employees to avoid rush hour travel. Home office was implemented, wherever possible. The measures were extended until April 26, 2020 after which the Federal Council followed a strategy to gradually emerge from the lockdown in three stages.

\emph{Phase 2 (relaxation):} The gradual easing of the enforced measures ended on June 6, 2020, when all events up to $300$ people and spontaneous gatherings for up to $30$ people were allowed again. High schools and universities were able to resume, all leisure and entertainment businesses as well as tourist attractions re-opened. Relatively calm summer months followed, suggesting that the spread of the virus might follow seasonal patterns.

\emph{Phase 3 (restriction):} Beginning October 19, 2020, a series of measures was introduced again as the situation worsened. Home office was recommended, culminating in a home office duty starting January 18, 2021. The duty was in place until June 26, 2021.

\emph{Phase 4 (relaxation):} However, even before abolishing the home office duty, other measures constraining the free movement of people were gradually eased as the third COVID wave was rather shallow. That is, while commutes were still reduced during phase 4, leisure activities were not. Again, the summer months were relatively calm.

\emph{Phase 5 (restriction):} With the onset of autumn end of September 2021, the last expansionary phase started. New, aggressive COVID variants emerged. As vaccination was fully rolled out with all the residents having had the opportunity to get vaccinated twice, a COVID certificate was required starting September 13, 2021. Political polarization was a consequence with society being divided into two segments with different constraints. However, home office became mandatory again for all, as of December 20, 2021 and was in place until February 3, 2022.

\emph{Phase 6 (relaxation):} Despite rocketing case numbers, the path back to normalization has started during February 2022 (as intensive care stations were not at limiting capacities any longer). Almost all measures have been abolished by February 17, 2022. As transition to a relatively normal live might have taken some time, we define the normalization phase to begin March 1, 2022.

The \emph{baseline} period is defined as all data points before October 31, 2019. The period from November, 2019 to the onset of the pandemic in late February, 2020 is discarded as a randomized controlled trial was conducted and mobility patterns are expected to be governed by the treatment.

\begin{table}[htbp]
\centering
\resizebox{\textwidth}{!}{%
\begin{tabular}{lllp{20em}l}
\toprule
Phase & Start & End & Description & Evolution\\
\midrule
1 & 2020-02-28 & 2020-04-26 & 1. Wave: Onset of the pandemic, high uncertainty, lockdown & Restriction\\
2 & 2020-04-27 & 2020-10-18 & Calm summer months & Relaxation\\
3 & 2020-10-19 & 2021-02-28 & 2. Wave: Home office recommendation followed by requirement (mid January) & Restriction\\
4 & 2021-03-01 & 2021-09-12 & 3. Wave: Shallow third wave, calm summer months & Relaxation\\
5 & 2021-09-13 & 2022-02-28 & 4. Wave: Spread of the Omicron variant, home office and certificate requirement & Restriction\\
6 & 2022-03-01 & 2022-06-29 & Normalization & Relaxation\\
\bottomrule
\caption{\label{tab:phases} The evolution of the pandemic in six phases.}
\end{tabular}
}%
\end{table}



\cleardoublepage%
\chapter{Sweave Chapter}
\label{ch:sweave-chapter}

\dictum[Daniel Heimgartner]{%
  All models are wrong - some models are wronger than others! }%
\vskip 1em

\begin{Schunk}
\begin{Sinput}
R> head(iris)
\end{Sinput}
\begin{Soutput}
  Sepal.Length Sepal.Width Petal.Length Petal.Width Species
1          5.1         3.5          1.4         0.2  setosa
2          4.9         3.0          1.4         0.2  setosa
3          4.7         3.2          1.3         0.2  setosa
4          4.6         3.1          1.5         0.2  setosa
5          5.0         3.6          1.4         0.2  setosa
6          5.4         3.9          1.7         0.4  setosa
\end{Soutput}
\end{Schunk}

\begin{table}
\centering
\begin{tabular}{|l|l|}
\hline
Command & Example \\
\hline
\verb|\proglang{}| & \proglang{R}, \proglang{C++} \\
\verb|\pkg{}| & \pkg{OPSR}, \pkg{OPSRtools} \\
\verb|\code{}| & \code{hello\_world}, \code{ys | yo ~ terms_s | terms_o} \\
\verb|\class{}| & \class{summary.opsr} \\
\verb|\fct{}| & \fct{hello\_world} \\
\hline
\end{tabular}
\caption{Some useful commands.}
\end{table}

% \cleardoublepage%
% \SweaveInput{}

\appendix
\cleardoublepage%
\def\dir{chapters/appendix}
\usepackage{natbib}
% citation style
\providecommand{\mybibstyle}{
  \setcitestyle{authoryear,round}
  \bibliographystyle{preamble/natbib/ivt-eng.bst}
  }
\mybibstyle


\cleardoublepage%
%********************************************************************
% Bibliography
%*******************************************************
% work-around to have small caps also here in the headline
\manualmark%
\markboth{\spacedlowsmallcaps{\bibname}}{\spacedlowsmallcaps{\bibname}} % work-around to have small caps also
%\phantomsection 
\refstepcounter{dummy}
\addtocontents{toc}{\protect\vspace{\beforebibskip}} % to have the bib a bit from the rest in the toc
\addcontentsline{toc}{chapter}{\tocEntry{\bibname}}
\label{app:bibliography}
{%
  \emergencystretch=1em%
  \printbibliography%
}

\bookmarksetup{startatroot}
%\pagenumbering{gobble}
\cleardoublepage%
%*******************************************************
% CV
%*******************************************************
\pdfbookmark[1]{Curriculum Vitae}{cv}

\bigskip

\begingroup
\let\clearpage\relax
\let\cleardoublepage\relax
\let\cleardoublepage\relax
\chapter*{Curriculum Vitae}

\section*{Personal data}

\noindent\cvleft{Name}%
\cvright{Albert Einstein}\\
%
\cvleft{Date of Birth}%
\cvright{March 14, 1879}\\
%
\cvleft{Place of Birth}%
\cvright{Ulm, Germany}\\
%
\cvleft{Citizen of}%
\cvright{Switzerland}

\section*{Education}

\noindent\cvleft{1896 --~1900}%
\cvright{%
  Eidgenössisches Polytechnikum, \\
  Zürich, Switzerland\\
  \emph{Final degree:} Diploma
}\vspace{0.75em}

\noindent\cvleft{1895 --~1896}%
\cvright{%
  Aargauische Kantonsschule (grammar school)\\
  Aarau, Switzerland \\
  \emph{Final degree:} Matura (university entrance diploma)
}\vspace{0.75em}

\noindent\cvleft{--~July 1894}%
\cvright{%
  Luitpold-Gymnasium (grammar school)\\
  Munich, Germany
}

\section*{Employment}

\noindent\cvleft{June 1902 --}%
\cvright{%
  Technical Expert, III Class\\
  \emph{Federal Office for Intellectual Property},\\ Bern, Switzerland
}\vspace{0.75em}

\endgroup
\cleardoublepage%
%*******************************************************
% Publications
%*******************************************************
\pdfbookmark[1]{Publications}{publications}
\chapter*{Publications}

\noindent
Articles in peer-reviewed journals:
\begin{refsection}[ownpubs]
  \small%
  \nocite{*}
  \printbibliography[heading=none,type=article]
\end{refsection}

\noindent
Conference contributions:
\begin{refsection}[ownpubs]
  \small%
  \nocite{*}
  \printbibliography[heading=none,type=inproceedings]
\end{refsection}

\end{document}
